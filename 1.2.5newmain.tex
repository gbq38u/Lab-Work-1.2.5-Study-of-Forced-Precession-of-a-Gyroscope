\documentclass[a4paper,10pt]{article} % Тип документа

% --- Пакеты и кодировки ---
\usepackage[left=3cm,right=3cm]{geometry}   % поля
\usepackage{float}                           % [H] для рисунков

\usepackage[utf8]{inputenc}                  % кириллица в исходнике (на новом LaTeX можно опустить)
\usepackage[T2A]{fontenc}                    % шрифтовая кодировка
\usepackage[russian]{babel}                  % локализация/переносы
\usepackage{csquotes}

% --- Графика (ВАЖНО: без dvips!) ---
\usepackage{graphicx}                        % Должен идти до \graphicspath
\graphicspath{{./}{pictures/}}               % ищем картинки и в текущей папке, и в pictures/
\DeclareGraphicsExtensions{.pdf,.png,.jpg}

% --- Математика ---
\usepackage{amsmath,amsfonts,amssymb,amsthm,mathtools}

% --- Ссылки/закладки ---
\usepackage{hyperref}                        % обычно подключают перед bookmark
\usepackage{bookmark}

% --- Границы плавающих объектов ---
\usepackage[section]{placeins}               % для \FloatBarrier

% --- Титул ---
\title{Отчёт о выполнении лабораторной работы\\
\textbf{Исследование вынужденной регулярной прецессии гироскопа}\\
\textbf{Лабораторная работа 1.2.5}}
\author{Г.А.Кузин }
\date{ Группа Б01-501 г. Долгопрудный, ФРКТ МФТИ, 09.10.2025 \\}

\begin{document}
\maketitle
\newpage

\section*{Аннотация}
Экспериментально исследована вынужденная регулярная прецессия гироскопа. Проведены измерения угловой скорости прецессии при различных моментах сил, определён момент инерции ротора методом крутильных колебаний и рассчитана частота его вращения. Полученные данные сопоставлены с теоретическими зависимостями, вытекающими из уравнений динамики твёрдого тела. Дополнительно выполнена оценка момента сил трения, влияющего на характер движения гироскопа. Экспериментальные результаты подтвердили линейную зависимость скорости прецессии от внешнего момента и позволили количественно учесть влияние трения.
\newpage
\section*{Введение}
Гироскоп представляет собой быстро вращающееся твёрдое тело, обладающее значительным моментом импульса. Благодаря этому моменту направление оси вращения гироскопа стремится сохраняться постоянным в пространстве при отсутствии внешних воздействий. Это свойство лежит в основе работы навигационных и стабилизирующих систем, используемых в авиации, космонавтике и мореплавании.

Однако в реальных условиях действие гироскопа не идеально: в осях подвеса и подшипниках неизбежно присутствуют силы трения, а также сопротивление воздуха, что приводит к постепенному изменению направления оси вращения. Эти эффекты вызывают прецессию — медленное вращение оси гироскопа под действием внешнего момента сил. В практических приборах такое явление проявляется как дрейф или погрешность показаний.

Экспериментальное исследование прецессии гироскопа позволяет не только подтвердить основные закономерности динамики твёрдого тела, но и количественно оценить влияние трения на устойчивость его работы. В данной лабораторной работе изучается вынужденная регулярная прецессия гироскопа, определяется момент инерции ротора, частота его вращения и оценивается вклад сил трения в отклонение реального движения от идеального.
\newpage
\section*{Методика эксперимента}
В процессе эксперимента используется гироскоп. Ротор гироскопа раскручивается электродвигателем до высокой угловой скорости $\omega$. К оси гироскопа прикладывается регулируемый момент силы $\vec{M}$, создаваемый системой грузов, подвешенных на рычаге. 

Для исследования вынужденной прецессии гироскопа можно раскрутить ротор до высокой угловой скорости $\omega$, что обеспечит значительный момент импульса 
\begin{equation}
    \vec{L} = I \vec{\omega},
    \label{eq:L}
\end{equation}
К оси гироскопа необходимо приложить внешний момент сил $\vec{M}$, в результате чего ось начнет совершать регулярную прецессию. Согласно уравнению динамики твёрдого тела 
\begin{equation}
    \frac{d\vec{L}}{dt} = \vec{M},
    \label{eq:M}
\end{equation}
угловая скорость прецессии $\Omega$ определяется соотношением
\begin{equation}
    \Omega = \frac{M}{I \omega}.
    \label{eq:Omega}
\end{equation}
Для проверки этого закона можно измерить скорость прецессии при различных значениях момента силы.  

Для нахождения момента инерции ротора необходимо применить метод крутильных колебаний: гироскоп подвешивается на нити, и измеряется период колебаний $T$, связанный с моментом инерции выражением
\begin{equation}
    T = 2\pi \sqrt{\frac{I}{f}},
    \label{eq:T}
\end{equation}
где $f$ — крутильная жёсткость нити.  

Частоту вращения ротора $\nu = \omega / 2\pi$ можно определить с помощью сравнения колебаний гироскопа с эталонным гармоническим сигналом, при котором на экране осциллографа наблюдались фигуры Лиссажу.  

Таким образом, методика включает раскрутку ротора, создание вынужденной прецессии, измерение её угловой скорости, определение момента инерции ротора методом крутильных колебаний и нахождение частоты его вращения по фигурам Лиссажу. Это позволяет экспериментально проверить зависимость $\Omega(M)$ и сопоставить результаты с теоретическими моделями.

\newpage
\section*{Результаты и их обсуждение}
В процессе эксперимента используется гироскоп. Ротор гироскопа раскручивается электродвигателем до высокой угловой скорости $\omega$. К оси гироскопа прикладывается регулируемый момент силы $\vec{M}$, создаваемый системой грузов, подвешенных на рычаге. 

Для исследования вынужденной прецессии гироскопа можно раскрутить ротор до высокой угловой скорости $\omega$, что обеспечит значительный момент импульса 
\begin{equation}
    \vec{L} = I \vec{\omega},
    \label{eq:L}
\end{equation}
К оси гироскопа необходимо приложить внешний момент сил $\vec{M}$, в результате чего ось начнет совершать регулярную прецессию. Согласно уравнению динамики твёрдого тела 
\begin{equation}
    \frac{d\vec{L}}{dt} = \vec{M},
    \label{eq:M}
\end{equation}
угловая скорость прецессии $\Omega$ определяется соотношением
\begin{equation}
    \Omega = \frac{M}{I \omega}.
    \label{eq:Omega}
\end{equation}
Для проверки этого закона можно измерить скорость прецессии при различных значениях момента силы.  

Для нахождения момента инерции ротора необходимо применить метод крутильных колебаний: гироскоп подвешивается на нити, и измеряется период колебаний $T$, связанный с моментом инерции выражением
\begin{equation}
    T = 2\pi \sqrt{\frac{I}{f}},
    \label{eq:T}
\end{equation}
где $f$ — крутильная жёсткость нити.  

Частоту вращения ротора $\nu = \omega / 2\pi$ можно определить с помощью сравнения колебаний гироскопа с эталонным гармоническим сигналом, при котором на экране осциллографа наблюдались фигуры Лиссажу.  

Таким образом, методика включает раскрутку ротора, создание вынужденной прецессии, измерение её угловой скорости, определение момента инерции ротора методом крутильных колебаний и нахождение частоты его вращения по фигурам Лиссажу. Это позволяет экспериментально проверить зависимость $\Omega(M)$ и сопоставить результаты с теоретическими моделями.

В ходе эксперимента была исследована зависимость угловой скорости регулярной прецессии гироскопа $\Omega$ от момента сил $M$, действующих на его ось. График зависимости приведён на рис.~1. Черные точки соответствуют экспериментальным данным, серые кресты обозначают доверительные интервалы, красная прямая иллюстрирует характерную линейную зависимость.

\begin{figure}[h!]
\centering
\includegraphics[width=0.75\textwidth]{gyro_graph_real_errors.png}
\caption{Зависимость угловой скорости прецессии $\Omega$ от момента силы $M$ с учётом погрешностей}
\end{figure}

Результаты эксперимента демонстрируют, что угловая скорость прецессии пропорциональна приложенному моменту силы, что соответствует теоретическому соотношению, приведённому в методике (см. формулу~(\ref{eq:Omega})).  
Полученная линейная зависимость подтверждает справедливость динамических уравнений для гироскопа (см. формулу~(\ref{eq:M})). Величины $\Omega$ изменяются в диапазоне от $0.035$~с$^{-1}$ при минимальном моменте $M = 0.068$~Н·м до $0.213$~с$^{-1}$ при максимальном моменте $M = 0.406$~Н·м. Из углового коэффициента графика была вычислена частота вращения ротора $\nu = (386 \pm 13)$~Гц, что согласуется с результатом, полученным по фигурам Лиссажу — $\nu = (395 \pm 15)$~Гц. Момент инерции ротора, найденный методом крутильных колебаний (см. формулу~(\ref{eq:T})), составил $I = (0.78 \pm 0.03)\cdot10^{-3}$~кг·м$^2$.  

Незначительные отклонения отдельных точек от прямой могут быть объяснены влиянием трения в осях карданового подвеса и погрешностями измерений времени. Таким образом, экспериментальные данные находятся в хорошем согласии с теоретической моделью.

Дополнительно в рамках эксперимента была выполнена оценка момента сил трения в осях гироскопа. Для этого после основной серии измерений наблюдалось поведение гироскопа при свободной прецессии без подвешенного груза. В этот момент на ось гироскопа действовал только момент сил трения в подшипниках и кардановом подвесе, вызывающий медленное опускание оси ротора.

Измерялось изменение угла наклона оси гироскопа $\Delta \alpha$ за время, соответствующее нескольким оборотам прецессии. Зная массу подвеса $m$, длину рычага $l$ и количество оборотов $N$, момент сил трения рассчитывался по формуле
\begin{equation}
    M_{\text{тр}} = \frac{m g l \Delta \alpha}{2\pi N}.
    \label{eq:Mtr}
\end{equation}
Этот метод позволяет количественно оценить потери энергии на трение в осях гироскопа и определить, насколько они влияют на характер регулярной прецессии. Полученные значения момента трения учитывались при анализе экспериментальных данных и при сравнении их с теоретической моделью.

Для оценки момента сил трения $M_{\text{тр}}$ в осях гироскопа, влияющего на движение гироскопа, была проведена серия измерений при различных моментах сил. Результаты представлены в таблице в приложении. Среднее значение момента трения составило
\[
M_{\text{тр}} = (1.37 \pm 0.26)\,\text{мН·м},
\]
что указывает на его заметное влияние на прецессию, особенно при малых моментах сил. Учёт этого фактора позволяет более точно описать динамику гироскопа и объяснить небольшие отклонения экспериментальных данных от идеальной линейной зависимости. 

Для анализа характера уменьшения частоты вращения ротора после выключения питания были нанесены экспериментальные точки и аппроксимирующая экспоненциальная кривая, описывающая затухание вращения по закону
\begin{equation}
    \nu(t) = \nu_0 e^{-t/\tau},
    \label{eq:decay}
\end{equation}
где $\nu_0$ — начальная частота вращения ротора, а $\tau$ — характерное время затухания, определяемое моментом сил трения.

\begin{figure}[h!]
\centering
\includegraphics[width=0.85\textwidth]{gyro_decay_with_exp.png}
\caption{Сравнение экспериментальных данных (синие точки) с экспоненциальной аппроксимацией (красная линия) при уменьшении частоты вращения гироскопа после выключения питания}
\label{fig:gyro_exp}
\end{figure}


На графике показана зависимость частоты вращения ротора гироскопа от времени после отключения питания. Сразу после выключения начальная частота составляла около $390$~Гц, далее наблюдается её плавное уменьшение по мере роста времени. Кривая имеет характер монотонного убывания, что соответствует постепенной потере энергии системы вследствие действия сил трения в подшипниках и сопротивления воздуха.  

На первых минутах скорость снижается сравнительно быстро, затем темп падения частоты становится более плавным. Такой характер свидетельствует о том, что в процессе торможения часть потерь компенсируется уменьшением сопротивлений при снижении скорости вращения. В результате ротор постепенно замедляется до значений порядка $150$~Гц через $\sim 20$ минут после выключения питания.  
Как видно из рис.~\ref{fig:gyro_exp}, экспериментальные данные хорошо описываются экспоненциальной зависимостью~(\ref{eq:decay}) при частотах выше $200$~Гц, что указывает на преобладание вязкого трения, пропорционального скорости вращения. При дальнейшем снижении частоты наблюдаются отклонения от идеальной экспоненты, обусловленные ростом доли сухого трения в подшипниках и неидеальностью подвеса. Таким образом, процесс торможения ротора носит в целом экспоненциальный характер с незначительными отклонениями на поздних этапах вращения.
\section*{Выводы}
Экспериментально подтверждена линейная зависимость угловой скорости прецессии гироскопа от приложенного момента сил, что согласуется с уравнением динамики твёрдого тела $\Omega = \frac{M}{I \omega}$ \cite{Sivuchin}. Определён момент инерции ротора $I = (0.78 \pm 0.03)\cdot10^{-3}$~кг·м$^2$, а частота его вращения по фигурам Лиссажу составила $\nu = (395 \pm 15)$~Гц, что совпадает с результатом, полученным из зависимости $\Omega(M)$, в пределах погрешности.  
Оценён момент сил трения $M_{\text{тр}} = (1.4 \pm 0.3)$~мН·м, который приводит к постепенному снижению частоты вращения и отклонению оси гироскопа. Полученные значения сопоставимы с данными лабораторного практикума \cite{Gladun}, что подтверждает корректность проведённого эксперимента и применимость классической теории прецессии к реальной установке.
Процесс уменьшения частоты вращения ротора после выключения питания имеет в основном экспоненциальный характер.  
Отклонения от идеальной зависимости при низких частотах объясняются возрастанием влияния сухого трения и неидеальностью подвеса.

\newpage
\section*{Приложение}
Формулы для оценки погрешностей:
В работе использовались следующие формулы для оценки погрешностей измерений.
Абсолютная погрешность момента силы:
\[
\Delta M = M \cdot \sqrt{\left(\frac{\Delta m}{m}\right)^2 + \left(\frac{\Delta l}{l}\right)^2},
\]
где $\Delta m$ — погрешность измерения массы груза, $\Delta l$ — погрешность измерения плеча.  

Абсолютная погрешность момента инерции ротора:
\[
\Delta I = I \cdot \sqrt{\left(\frac{\Delta T}{T}\right)^2 + \left(\frac{\Delta I_\text{эт}}{I_\text{эт}}\right)^2},
\]
где $T$ — период колебаний, $I_\text{эт}$ — момент инерции эталонного цилиндра.  

Абсолютная погрешность частоты вращения:
\[
\Delta \nu = \nu \cdot \sqrt{\left(\frac{\Delta a}{a}\right)^2 + \left(\frac{\Delta I}{I}\right)^2},
\]
где $a$ — наклон графика $\Omega(M)$.
В этой таблице приведены экспериментальные данные и рассчитанные моменты сил трения для различных масс грузов, используемых для создания момента силы на оси гироскопа.
\begin{table}[h!]
\centering
\caption{Экспериментальные данные и оценка момента сил трения}
\begin{tabular}{|c|c|c|c|}
\hline
$m$, г & $\Omega$, 1/с & $M$, Н·м & $M_{\text{тр}}$, мН·м \\ \hline
57  & 0.035 & 0.0677 & 1.65 \\
92  & 0.057 & 0.109  & 1.46 \\
116 & 0.072 & 0.137  & 1.50 \\
142 & 0.088 & 0.169  & 1.42 \\
180 & 0.112 & 0.213  & 1.40 \\
219 & 0.135 & 0.260  & 1.36 \\
273 & 0.169 & 0.325  & 1.37 \\
341 & 0.213 & 0.406  & 1.38 \\
74  & 0.046 & 0.088  & 1.39 \\ \hline
\end{tabular}
\end{table}

\subsection*{Схема экспериментальной установки}

Экспериментальная установка представляет собой гироскоп, закреплённый в кардановом подвесе, что обеспечивает свободу его оси в пространстве.  
К оси гироскопа прикладывается регулируемый момент силы, создаваемый системой грузов.  
Для измерения угловой скорости прецессии используется шкала или система оптической регистрации.  

Ротор гироскопа раскручивается электродвигателем, питаемым от источника постоянного тока.  
Для определения частоты вращения применяется метод фигур Лиссажу с использованием осциллографа.  
Момент инерции ротора определяется методом крутильных колебаний при подвесе на тонкой упругой нити.  

На рисунках ниже приведены общие схемы установки, измерительных элементов и чертеж с расставленными силами:

\begin{center}
\includegraphics[width=0.6\textwidth]{img1.png} \\[1ex]
\includegraphics[width=0.6\textwidth]{img2.png} \\[1ex]
\includegraphics[width=0.6\textwidth]{img3.png}
\end{center}

Таким образом, установка позволяет экспериментально определить момент инерции ротора, его частоту вращения, а также исследовать зависимость угловой скорости прецессии от приложенного момента сил.
\newpage
\section*{Литература}
\begin{thebibliography}{99}
\bibitem{Gladun}
Д.\,А. Гладун, \textit{Лабораторный практикум по общей физике}. — М.: Издательство, 2012.

\bibitem{Sivuchin}
Д.\,В. Сивухин, \textit{Общий курс физики. Механика. Т.~1}. — ФИЗМАТЛИТ, 2016.
\end{thebibliography}

\end{document}
